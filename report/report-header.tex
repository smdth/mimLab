%%% Special LaTeX headers
%
% Remove and edit as desired. Simply serves as an example.
% Here, reduce excessive use of whitespace a little bit.

\usepackage{lastpage}
\usepackage{amsmath, amsthm, amssymb, amsfonts}
\usepackage{pifont}
\usepackage{mathrsfs}
\usepackage{bbm}
\usepackage{stmaryrd}
\usepackage{wasysym}
\usepackage{tikz}
\usetikzlibrary{shapes, positioning, arrows, automata, fit}
%\usepackage{wrapfig}

% loop instead of for
%\SetKwFor{For}{\textbf{loop}}{do}{end loop}

\setlength{\parindent} {0pt}
\setlength{\parskip}{0.5ex}

\usepackage{charter}
\usepackage[bitstream-charter]{mathdesign}
%\usepackage{fancyhdr, hyperref, titlesec, blindtext, color}
\usepackage{fancyhdr, hyperref, titlesec, color}
\hypersetup{hidelinks, linktoc=all}

\usepackage{xcolor}


\lstset{
    basicstyle=\ttfamily,
    numbers=left,
    keywordstyle=\color[rgb]{0.13,0.29,0.53}\bfseries,
    stringstyle=\color[rgb]{0.31,0.60,0.02},
    commentstyle=\color[rgb]{0.56,0.35,0.01}\itshape,
    numberstyle=\footnotesize,
    stepnumber=1,
    numbersep=5pt,
    backgroundcolor=\color[RGB]{248,248,248},
    showspaces=false,
    showstringspaces=false,
    showtabs=false,
    tabsize=2,
    captionpos=b,
    breaklines=true,
    breakatwhitespace=true,
    breakautoindent=true,
    escapeinside={\%*}{*)},
    linewidth=40em,
    basewidth=0.5em,
}

\newcounter{points}

\pagestyle{fancy}
\fancyhf{}

\titleformat{\section}[hang]{\fontfamily{qhv}}{\fontfamily{qhv}\bfseries\thesection\hsp\textcolor{gray75}{$\vert$}\hsp}{0pt}{\Large\bfseries}

\newcommand{\uebkopfzeile}[4]{
	\begin{center}
		\begin{bf}
			\usefont{T1}{qhv}{b}{n}
			#1 \hfill #2\\
			#3 \hfill #4\\
		\end{bf}
	\end{center}  
}

\newcommand{\uebtitel}[2]{
	\begin{center}
		\begin{bf}
			\usefont{T1}{qhv}{b}{n}
			\bigskip
			{\LARGE #1}\\
		\end{bf}
		\textit{\rmfamily(Deadline: #2)}
	\end{center}
}

\makeatletter
\newcommand{\solution}{\@ifstar{\uebungsblatt@zsolution}{\uebungsblatt@solution}}
\newcommand{\uebungsblatt@zsolution}[3]{%
	\begin{center}
		\begin{bf}
			#1: {\it #2}\hfill [ \quad ] of #3 Points\\
		\end{bf}
	\end{center}
}
\newcommand{\uebungsblatt@solution}[3]{\begin{center}
		\begin{bf}
			{\usefont{T1}{qhv}{b}{n}Solution of Exercise #1:} 
			\parbox[t]{,4\textwidth}{{\it #2}}\hfill{\usefont{T1}{qhv}{b}{n} [ 
				\quad ] of #3 Points}\\
		\end{bf}
	\end{center}
	\addtocounter{points}{#3}
}
\makeatother

% sets
\newcommand*{\F}{\mathbb F}
\newcommand*{\N}{\mathbb N}
\newcommand*{\Z}{\mathbb Z}
\newcommand*{\Q}{\mathbb Q}
\newcommand*{\R}{\mathbb R}
\renewcommand*{\C}{\mathbb C}
\newcommand*{\Pow}{\mathcal P}
\newcommand*{\K}{\mathbb K}

% operators
\newcommand*{\et}{\mathbin{\&}}
\DeclareMathOperator{\com}{com}
\DeclareMathOperator{\im}{im}
\DeclareMathOperator{\rp}{Re}
\DeclareMathOperator{\ip}{Im}
\DeclareMathOperator{\lin}{span}
\DeclareMathOperator{\row}{row}
\DeclareMathOperator{\supp}{supp}
\DeclareMathOperator{\ve}{vec}
%\DeclareMathOperator{\co}{coord}
\DeclareMathOperator{\id}{id}
\DeclareMathOperator{\sgn}{sgn}
%\DeclareMathOperator{\det}{det}
\DeclareMathOperator{\rank}{rank}
\DeclareMathOperator{\rk}{rk}
\DeclareMathOperator{\End}{End}
\DeclareMathOperator{\Hom}{Hom}
\DeclareMathOperator{\Ker}{Ker}
\DeclareMathOperator{\Tr}{Tr}
\DeclareMathOperator{\Eig}{Eig}
\DeclareMathOperator{\Abb}{Abb}
\DeclareMathOperator{\Bil}{Bil}
\DeclareMathOperator{\Mat}{Mat}
\DeclareMathOperator{\ord}{ord}

\renewcommand\over[2]{\genfrac{}{}{0pt}{}{#1}{#2}}

% symbols
\renewcommand{\i}{\mathring{\imath}}
\renewcommand{\j}{\mathring{\jmath}}
\newcommand*{\ph}{\varphi}
\newcommand*{\ps}{\psi}
\newcommand*{\la}{\lambda}
\newcommand*{\e}{\varepsilon}
\renewcommand*{\a}{\alpha}
\renewcommand*{\b}{\beta}
\newcommand{\si}{\sigma}
\newcommand{\ta}{\tau}
\newcommand{\rh}{\varrho}
\newcommand{\de}{\delta}
\newcommand{\Ph}{\Phi}
\newcommand{\ze}{\zeta}
\newcommand{\ch}{\chi}
\newcommand{\al}{\alpha}
\newcommand{\be}{\beta}
\newcommand{\idleq}{\trianglelefteq}

\newcommand*{\skal}[1]{\left\langle #1\right\rangle}

% changes \qed to a filled square
\renewcommand*{\qedsymbol}{\ensuremath{\blacksquare}}
% set mathcal as it is vanilla, I like it more like that than the rsf thingy
\SetMathAlphabet{\mathcal}{normal}{OMS}{cmsy}{m}{bf}

% strucutres
\makeatletter
\renewcommand*{\vec}{\@ifstar{\cscallr@ovec}{\cscallr@uvec}}
\newcommand*{\cscallr@ovec}[1]{\overrightarrow{#1}}
\newcommand*{\cscallr@uvec}[1]{\text{vec}_{\underline{#1}}}
\makeatother
\newcommand*{\coord}[1]{\text{coord}_{\underline{#1}}}
\newcommand*\otilde[1]{\widetilde{#1}}
\renewcommand*{\o}[1]{\overline{#1}}
\makeatletter
% \vor for short form, \vor* for long form
\newcommand*{\vor}{\@ifstar{\cscallr@lvor}{\cscallr@svor}}
\newcommand*{\cscallr@svor}{\textbf{Vor.:} }
\newcommand*{\cscallr@lvor}{\textbf{Voraussetzung:} }
% \beh, see \vor
\newcommand*{\beh}{\@ifstar{\cscallr@lbeh}{\cscallr@sbeh}}
\newcommand*{\cscallr@sbeh}{\textbf{Beh.:} }
\newcommand*{\cscallr@lbeh}{\textbf{Behauptung:} }
% \bew, see \vor	
\newcommand*{\bew}{\@ifstar{\cscallr@lbew}{\cscallr@sbew}}
\newcommand*{\cscallr@sbew}{\textit{Bew.:} }
\newcommand*{\cscallr@lbew}{\textit{Beweis:} }
\makeatother

\newcommand*{\zz}{Zu zeigen: }
\makeatletter
\newcommand{\arr}{\@ifstar{\cscallr@larr}{\cscallr@arr}}
\newcommand\cscallr@larr[1]{\left\{\begin{array}{c} #1 \end{array}\right.}
\newcommand\cscallr@arr[1]{\left\{\begin{array}{c} #1 \end{array}\right\}}
\makeatother
\renewcommand*\u[1]{\underset{#1}}
\newcommand\lr[1]{\left\{ #1 \right\}}
%#1.2 ?
% links to \label{key}
\newcommand*\rref[1]{[$\rightarrow$\hyperref[#1]{#1}]} % with brackets
\newcommand*\bref[2]{[$\rightarrow$\hyperref[#1]{#1},$\rightarrow$\hyperref[#2]{#2}]}
% dual, with brackets
\newcommand*\hrref[1]{\hyperref[#1]{#1}} % without brackets, for refs to (a)
%for example
% for normal column vectors use cvec, for small ones (preferably in text) use
%cvec*
\makeatletter
\newcommand{\cvec}{\@ifstar{\cscallr@scvec}{\cscallr@cvec}}
\newcommand{\cscallr@scvec}[1]{%
	\text{$\left(\begin{smallmatrix}#1\end{smallmatrix}\right)$}
}
\newcommand{\cscallr@cvec}[1]{\begin{pmatrix}#1\end{pmatrix}}
\makeatother
\newcommand{\dvec}[1]{\begin{vmatrix}#1\end{vmatrix}}
\newcommand{\vvec}[2]{ \left(\begin{array}{#1}#2\end{array}\right) }
\newcommand*{\smallfrac}[2]{\frac{_{#1}}{^{#2}}}
\newcommand{\bvec}[1]{\begin{bmatrix}#1\end{bmatrix}}

\fancyhead[R]{Page: \thepage \hspace{0.5ex} of \pageref{LastPage}}
\fancyhead[L]{Joe O'Grady, Thomas Schmidt}

\fancypagestyle{firststyle}
{
   \fancyhf{}
\fancyhead[R]{Page: \thepage \hspace{0.5ex} of \pageref{LastPage}}
\fancyhead[L]{Joe O'Grady, Thomas Schmidt}
}
%Kopfzeile auf jeder Seite mit eurem Namen und Seitenzahl

\thispagestyle{firststyle}
