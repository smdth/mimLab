\documentclass[10pt,a4paper]{scrartcl}
\usepackage{lmodern}
\usepackage{amssymb,amsmath}
\usepackage{ifxetex,ifluatex}
\usepackage{fixltx2e} % provides \textsubscript
\ifnum 0\ifxetex 1\fi\ifluatex 1\fi=0 % if pdftex
  \usepackage[T1]{fontenc}
  \usepackage[utf8]{inputenc}
\else % if luatex or xelatex
  \ifxetex
    \usepackage{mathspec}
    \usepackage{xltxtra,xunicode}
  \else
    \usepackage{fontspec}
  \fi
  \defaultfontfeatures{Mapping=tex-text,Scale=MatchLowercase}
  \newcommand{\euro}{€}
\fi
% use upquote if available, for straight quotes in verbatim environments
\IfFileExists{upquote.sty}{\usepackage{upquote}}{}
% use microtype if available
\IfFileExists{microtype.sty}{%
\usepackage{microtype}
\UseMicrotypeSet[protrusion]{basicmath} % disable protrusion for tt fonts
}{}
\ifxetex
  \usepackage[setpagesize=false, % page size defined by xetex
              unicode=false, % unicode breaks when used with xetex
              xetex]{hyperref}
\else
  \usepackage[unicode=true]{hyperref}
\fi
\usepackage[usenames,dvipsnames]{color}
\hypersetup{breaklinks=true,
            bookmarks=true,
            pdfauthor={Joe O'Grady, Thomas Schmidt},
            pdftitle={Machines in Motion Lab Report},
            colorlinks=true,
            citecolor=blue,
            urlcolor=blue,
            linkcolor=magenta,
            pdfborder={0 0 0}}
\urlstyle{same}  % don't use monospace font for urls
\usepackage{listings}
\setlength{\parindent}{0pt}
\setlength{\parskip}{6pt plus 2pt minus 1pt}
\setlength{\emergencystretch}{3em}  % prevent overfull lines
\providecommand{\tightlist}{%
  \setlength{\itemsep}{0pt}\setlength{\parskip}{0pt}}
\setcounter{secnumdepth}{0}

\title{Machines in Motion Lab Report}
\author{Joe O'Grady, Thomas Schmidt}
\newcommand\thedeadline{2016-01-11}
\date{\today}
%%% Special LaTeX headers
%
% Remove and edit as desired. Simply serves as an example.
% Here, reduce excessive use of whitespace a little bit.

\usepackage{lastpage}
\usepackage{amsmath, amsthm, amssymb, amsfonts}
\usepackage{pifont}
\usepackage{mathrsfs}
\usepackage{bbm}
\usepackage{stmaryrd}
\usepackage{wasysym}
\usepackage{tikz}
\usetikzlibrary{shapes, positioning, arrows, automata, fit}
%\usepackage{wrapfig}

% loop instead of for
%\SetKwFor{For}{\textbf{loop}}{do}{end loop}

\setlength{\parindent} {0pt}
\setlength{\parskip}{0.5ex}

\usepackage{charter}
\usepackage[bitstream-charter]{mathdesign}
%\usepackage{fancyhdr, hyperref, titlesec, blindtext, color}
\usepackage{fancyhdr, hyperref, titlesec, color}
\hypersetup{hidelinks, linktoc=all}

\usepackage{xcolor}


\lstset{
    basicstyle=\ttfamily,
    numbers=left,
    keywordstyle=\color[rgb]{0.13,0.29,0.53}\bfseries,
    stringstyle=\color[rgb]{0.31,0.60,0.02},
    commentstyle=\color[rgb]{0.56,0.35,0.01}\itshape,
    numberstyle=\footnotesize,
    stepnumber=1,
    numbersep=5pt,
    backgroundcolor=\color[RGB]{248,248,248},
    showspaces=false,
    showstringspaces=false,
    showtabs=false,
    tabsize=2,
    captionpos=b,
    breaklines=true,
    breakatwhitespace=true,
    breakautoindent=true,
    escapeinside={\%*}{*)},
    linewidth=40em,
    basewidth=0.5em,
}

\newcounter{points}

\pagestyle{fancy}
\fancyhf{}

\titleformat{\section}[hang]{\fontfamily{qhv}}{\fontfamily{qhv}\bfseries\thesection\hsp\textcolor{gray75}{$\vert$}\hsp}{0pt}{\Large\bfseries}

\newcommand{\uebkopfzeile}[4]{
	\begin{center}
		\begin{bf}
			\usefont{T1}{qhv}{b}{n}
			#1 \hfill #2\\
			#3 \hfill #4\\
		\end{bf}
	\end{center}  
}

\newcommand{\uebtitel}[2]{
	\begin{center}
		\begin{bf}
			\usefont{T1}{qhv}{b}{n}
			\bigskip
			{\LARGE #1}\\
		\end{bf}
		\textit{\rmfamily(Deadline: #2)}
	\end{center}
}

\makeatletter
\newcommand{\solution}{\@ifstar{\uebungsblatt@zsolution}{\uebungsblatt@solution}}
\newcommand{\uebungsblatt@zsolution}[3]{%
	\begin{center}
		\begin{bf}
			#1: {\it #2}\hfill [ \quad ] of #3 Points\\
		\end{bf}
	\end{center}
}
\newcommand{\uebungsblatt@solution}[3]{\begin{center}
		\begin{bf}
			{\usefont{T1}{qhv}{b}{n}Solution of Exercise #1:} 
			\parbox[t]{,4\textwidth}{{\it #2}}\hfill{\usefont{T1}{qhv}{b}{n} [ 
				\quad ] of #3 Points}\\
		\end{bf}
	\end{center}
	\addtocounter{points}{#3}
}
\makeatother

% sets
\newcommand*{\F}{\mathbb F}
\newcommand*{\N}{\mathbb N}
\newcommand*{\Z}{\mathbb Z}
\newcommand*{\Q}{\mathbb Q}
\newcommand*{\R}{\mathbb R}
\renewcommand*{\C}{\mathbb C}
\newcommand*{\Pow}{\mathcal P}
\newcommand*{\K}{\mathbb K}

% operators
\newcommand*{\et}{\mathbin{\&}}
\DeclareMathOperator{\com}{com}
\DeclareMathOperator{\im}{im}
\DeclareMathOperator{\rp}{Re}
\DeclareMathOperator{\ip}{Im}
\DeclareMathOperator{\lin}{span}
\DeclareMathOperator{\row}{row}
\DeclareMathOperator{\supp}{supp}
\DeclareMathOperator{\ve}{vec}
%\DeclareMathOperator{\co}{coord}
\DeclareMathOperator{\id}{id}
\DeclareMathOperator{\sgn}{sgn}
%\DeclareMathOperator{\det}{det}
\DeclareMathOperator{\rank}{rank}
\DeclareMathOperator{\rk}{rk}
\DeclareMathOperator{\End}{End}
\DeclareMathOperator{\Hom}{Hom}
\DeclareMathOperator{\Ker}{Ker}
\DeclareMathOperator{\Tr}{Tr}
\DeclareMathOperator{\Eig}{Eig}
\DeclareMathOperator{\Abb}{Abb}
\DeclareMathOperator{\Bil}{Bil}
\DeclareMathOperator{\Mat}{Mat}
\DeclareMathOperator{\ord}{ord}

\renewcommand\over[2]{\genfrac{}{}{0pt}{}{#1}{#2}}

% symbols
\renewcommand{\i}{\mathring{\imath}}
\renewcommand{\j}{\mathring{\jmath}}
\newcommand*{\ph}{\varphi}
\newcommand*{\ps}{\psi}
\newcommand*{\la}{\lambda}
\newcommand*{\e}{\varepsilon}
\renewcommand*{\a}{\alpha}
\renewcommand*{\b}{\beta}
\newcommand{\si}{\sigma}
\newcommand{\ta}{\tau}
\newcommand{\rh}{\varrho}
\newcommand{\de}{\delta}
\newcommand{\Ph}{\Phi}
\newcommand{\ze}{\zeta}
\newcommand{\ch}{\chi}
\newcommand{\al}{\alpha}
\newcommand{\be}{\beta}
\newcommand{\idleq}{\trianglelefteq}

\newcommand*{\skal}[1]{\left\langle #1\right\rangle}

% changes \qed to a filled square
\renewcommand*{\qedsymbol}{\ensuremath{\blacksquare}}
% set mathcal as it is vanilla, I like it more like that than the rsf thingy
\SetMathAlphabet{\mathcal}{normal}{OMS}{cmsy}{m}{bf}

% strucutres
\makeatletter
\renewcommand*{\vec}{\@ifstar{\cscallr@ovec}{\cscallr@uvec}}
\newcommand*{\cscallr@ovec}[1]{\overrightarrow{#1}}
\newcommand*{\cscallr@uvec}[1]{\text{vec}_{\underline{#1}}}
\makeatother
\newcommand*{\coord}[1]{\text{coord}_{\underline{#1}}}
\newcommand*\otilde[1]{\widetilde{#1}}
\renewcommand*{\o}[1]{\overline{#1}}
\makeatletter
% \vor for short form, \vor* for long form
\newcommand*{\vor}{\@ifstar{\cscallr@lvor}{\cscallr@svor}}
\newcommand*{\cscallr@svor}{\textbf{Vor.:} }
\newcommand*{\cscallr@lvor}{\textbf{Voraussetzung:} }
% \beh, see \vor
\newcommand*{\beh}{\@ifstar{\cscallr@lbeh}{\cscallr@sbeh}}
\newcommand*{\cscallr@sbeh}{\textbf{Beh.:} }
\newcommand*{\cscallr@lbeh}{\textbf{Behauptung:} }
% \bew, see \vor	
\newcommand*{\bew}{\@ifstar{\cscallr@lbew}{\cscallr@sbew}}
\newcommand*{\cscallr@sbew}{\textit{Bew.:} }
\newcommand*{\cscallr@lbew}{\textit{Beweis:} }
\makeatother

\newcommand*{\zz}{Zu zeigen: }
\makeatletter
\newcommand{\arr}{\@ifstar{\cscallr@larr}{\cscallr@arr}}
\newcommand\cscallr@larr[1]{\left\{\begin{array}{c} #1 \end{array}\right.}
\newcommand\cscallr@arr[1]{\left\{\begin{array}{c} #1 \end{array}\right\}}
\makeatother
\renewcommand*\u[1]{\underset{#1}}
\newcommand\lr[1]{\left\{ #1 \right\}}
%#1.2 ?
% links to \label{key}
\newcommand*\rref[1]{[$\rightarrow$\hyperref[#1]{#1}]} % with brackets
\newcommand*\bref[2]{[$\rightarrow$\hyperref[#1]{#1},$\rightarrow$\hyperref[#2]{#2}]}
% dual, with brackets
\newcommand*\hrref[1]{\hyperref[#1]{#1}} % without brackets, for refs to (a)
%for example
% for normal column vectors use cvec, for small ones (preferably in text) use
%cvec*
\makeatletter
\newcommand{\cvec}{\@ifstar{\cscallr@scvec}{\cscallr@cvec}}
\newcommand{\cscallr@scvec}[1]{%
	\text{$\left(\begin{smallmatrix}#1\end{smallmatrix}\right)$}
}
\newcommand{\cscallr@cvec}[1]{\begin{pmatrix}#1\end{pmatrix}}
\makeatother
\newcommand{\dvec}[1]{\begin{vmatrix}#1\end{vmatrix}}
\newcommand{\vvec}[2]{ \left(\begin{array}{#1}#2\end{array}\right) }
\newcommand*{\smallfrac}[2]{\frac{_{#1}}{^{#2}}}
\newcommand{\bvec}[1]{\begin{bmatrix}#1\end{bmatrix}}

\fancyhead[R]{Page: \thepage \hspace{0.5ex} of \pageref{LastPage}}
\fancyhead[L]{Joe O'Grady, Thomas Schmidt}

\fancypagestyle{firststyle}
{
   \fancyhf{}
\fancyhead[R]{Page: \thepage \hspace{0.5ex} of \pageref{LastPage}}
\fancyhead[L]{Joe O'Grady, Thomas Schmidt}
}
%Kopfzeile auf jeder Seite mit eurem Namen und Seitenzahl

\thispagestyle{firststyle}

% Redefines (sub)paragraphs to behave more like sections
\ifx\paragraph\undefined\else
\let\oldparagraph\paragraph
\renewcommand{\paragraph}[1]{\oldparagraph{#1}\mbox{}}
\fi
\ifx\subparagraph\undefined\else
\let\oldsubparagraph\subparagraph
\renewcommand{\subparagraph}[1]{\oldsubparagraph{#1}\mbox{}}
\fi

\begin{document}
\begin{center}
	\begin{bf}
		\usefont{T1}{qhv}{b}{n}
		\bigskip
		{\LARGE Machines in Motion Lab Report}\\
	\end{bf}
	\textit{\rmfamily(Deadline: 2016-01-11)}
\end{center}

\section{Before the lab}\label{before-the-lab}

\subsection{A.}\label{a.}

We wrote the following python class, called \lstinline!mimLocator!

\begin{lstlisting}[language=Python]
#!/usr/bin/env python2.7

import numpy as np
import sys

class Locator:
    def __init__(self, p1, p2, p3):

        # initialize points
        self.p1 = np.array(p1)
        self.p2 = np.array(p2)
        self.p3 = np.array(p3)

        # set vectors form p1 to p2 and from p1 to p3
        self.v12 = self.p2 - self.p1
        self.v13 = self.p3 - self.p1

        self.v12 = self.v12 / np.linalg.norm(self.v12)
        # calculate the normal to v12 and v13
        self.n = np.cross(self.v12, self.v13)
        self.n = self.n / np.linalg.norm(self.n)

        # calculate corresponding axes of the plane, having p1 as base
        # axisX = v12 / |v12|
        self.axisX = self.v12 / np.linalg.norm(self.v12)
        self.axisY = np.cross(self.n, self.axisX)
        #self.axisY = self.axisY / np.linalg.norm(self.axisY)

        self.rot = np.matrix([np.array(self.axisX), np.array(self.axisY),
                              np.array(self.n)])

        self.q = [ 0,0,0,0 ]
        self.q[3] = np.sqrt(1 + self.rot[0,0] + self.rot[1,1] \
                + self.rot[2,2]) / 2
        self.q[0] = (self.rot[2,1] - self.rot[1,2]) / (4 * self.q[3])
        self.q[1] = (self.rot[0,2] - self.rot[2,0]) / (4 * self.q[3])
        self.q[2] = (self.rot[1,0] - self.rot[0,1]) / (4 * self.q[3])

        self.scale = np.linalg.norm(self.p2 - self.p1)

    def planeToCartesian(self, x, y):
        """Takes a point (x,y) on the plane the Locator was initialized for and
        transforms it into (x,y,z) coordinates located in cartesian space.
        """
        p = self.p1 + (x * self.scale * self.axisX) \
                + (y * self.scale * self.axisY)
        return p

def main():
    l = Locator([float(sys.argv[1]), float(sys.argv[2]),
        float(sys.argv[3])], [float(sys.argv[4]), float(sys.argv[5]),
        float(sys.argv[6])], [float(sys.argv[7]), float(sys.argv[8]),
        float(sys.argv[9])])
    print l.planeToCartesian(sys.argv[10], sys.argv[11])

if __name__ == '__main__':
    main()
\end{lstlisting}

In the \lstinline!__init__! function, we initialise the class variables
and calculate the normal vector, the rotation matrix, the quaternions,
etc.. The \lstinline!planeToCartesian! function takes a point in
2d-space and transforms it into the corresponding point on the plane.
The \lstinline!main! function is to make scripting for later tasks
easier. It lets the script take three points on the plane as argument,
from which the corresponding plane is calculated, and then for another
2d-point, that is handed over as commandline argument as well, returns
its output if given to \lstinline!planeToCartesian!.

\subsection{C.}\label{c.}

We amended the given code in the indicated area as follows:

\begin{lstlisting}[language=Python]
#!/usr/bin/env python

# This script should return the x y z and orientation coordinates of the end effector of the left limb.
# PLEASE ADD YOUR CODE WHERE INDICATED
# Avoid modifying the rest of the code if not necessary
#
# Authors: Stefano Pietrosanti - s.pietrosanti@pgr.reading.ac.uk
#          Guy Butcher

import rospy
import baxter_interface
import numpy
from geometry_msgs.msg import (
    PoseStamped,
    Pose,
    Point,
    Quaternion,
)

print("MIM tutorial: forward kinematics.")
# Initialising ROS node
rospy.init_node("SSE_forward_kinematics")

######################  INSERT YOUR CODE HERE
# Create a "Limb" instance called "left_arm" linked to Baxter's left limb

left_arm = baxter_interface.Limb('left')

# Create a "pose" variable which holds the output of endpoint_pose()

pose = left_arm.endpoint_pose()

######################



# Return pose
print("Endpoint coordinates:")
print("X: " + str(pose['position'].x))
print("Y: " + str(pose['position'].y))
print("Z: " + str(pose['position'].z))
\end{lstlisting}

\section{During The Lab}\label{during-the-lab}

\subsection{A.}\label{a.-1}

\subsection{B.}\label{b.}

\subsection{C.}\label{c.-1}

\subsection{D.}\label{d.}

\subsection{E.}\label{e.}

\subsection{F.}\label{f.}

\subsection{G.}\label{g.}

\subsection{H.}\label{h.}

\subsection{I.}\label{i.}

\subsection{J.}\label{j.}

\subsection{K.}\label{k.}

\subsection{L.}\label{l.}

\end{document}
